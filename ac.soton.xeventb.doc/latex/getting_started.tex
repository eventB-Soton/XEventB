\section{Getting Started}
\label{sec:getting-started}

\subsection{Installation}
\label{sec:installation}

\subsubsection{Setup}
\label{sec:setup}

\begin{itemize}
\item Before installing the XEvent-B Front-end feature, you need to add the XText update site (\url{http://download.eclipse.org/modeling/tmf/xtext/updates/composite/releases/}) as an additional software site (see Figure~\ref{fig:xtext-updatesite}).
\begin{figure}[!htbp]
  \centering
  \ifplastex
  \includegraphics[width=512]{figures/XTextUpdateSite}
  \else
  \includegraphics[width=0.9\textwidth]{figures/XTextUpdateSite}
  \fi
  \caption{Adding XText Update Site}
  \label{fig:xtext-updatesite}
\end{figure}


\item The XEvent-B front-end feature is available from the main Rodin update site (under ``Modelling Extensions'' category). There are two versions of the feature, \emph{eXtended Event-B (XEvent-B)} providing facilities for working with XEvent-B Front-end, and the \emph{eXtended Event-B (XEvent-B) (SDK)} is the feature including source code for software developers (see Figure~\ref{fig:EventBXText-installation}).
\begin{figure}[!htbp]
  \centering
  \ifplastex
  \includegraphics[width=512]{figures/EventBXTextInstallation}
  \else
  \includegraphics[width=0.9\textwidth]{figures/EventBXTextInstallation}
  \fi
  \caption{Adding XText Update Site}
  \label{fig:EventBXText-installation}
\end{figure}

\end{itemize}

\subsubsection{Release Notes}
\label{sec:release-notes}
\paragraph{0.0.7}
\begin{itemize}
	\item Fixed Issues:
	\begin{itemize}
		\item Issue \#8: Comments are not parsed.
		\item Issue \#10: Variants not translated: Fix is part of inclusion plug-in release 0.2.0.
	\end{itemize}
	\item Enhancement: (Issue \#11)
	\begin{itemize}
		\item Machines from different projects can now be included.
		\item Machines are now included using qualified name defined as: projectName.machineName
	\end{itemize}
	\item Order of elements generation is updated:
	\begin{itemize}
		\item flattened machines now have the included machine elements generated before the source machine.
		\item Order of generating elements of multiple inclusions and/or instances is from last to first.
		\item This update is part of inclusion plug-in release 0.2.0.
	\end{itemize}
\end{itemize}
\paragraph{0.0.6}
\begin{itemize}
\item Renamed plug-ins and features to XEvent-B (instead of Event-B XText).
\item XEvent-B Branding (0.0.2): Renamed from Event-B XText Branding.
\item XEvent-B Documentations (0.0.2): Renamed from Event-B XText Documentations.
\item XEvent-B Cheatsheets (0.0.2): Renamed from Event-B XText Cheatsheets.
\item XEvent-B Common (0.0.3): Renamed from Event-B XText Common.
\item XEvent-B UI (0.0.2): Renamed from Event-B XText UI.
\item XEvent-B XContext (0.0.4): Renamed from Event-B XText Context.
\item XEvent-B XContext IDE (0.0.3): Renamed from Event-B XText Context IDE.
\item XEvent-B XContext UI (0.0.3): Renamed from Event-B Context UI.
\item XEvent-B XMachine (0.0.4): Renamed from Event-B XText Machine.
	\begin{itemize}
		\item Support Machine Inclusion and Event Synchronisation.
	\end{itemize}	
\item XEvent-B XMachine IDE (0.0.3): Renamed from Event-B XText Machine IDE.
\item XEvent-B XMachine UI (0.0.3): Renamed from Event-B XText Machine UI.
\end{itemize}

\paragraph{0.0.5}
\begin{itemize}
\item Event-B XText Documentations (0.0.1): Documentation plug-in (Initial version).
\end{itemize}

\paragraph{0.0.4}
\begin{itemize}
\item Updated plug-in dependency for the feature
\end{itemize}

\paragraph{0.0.3}
\begin{itemize}
\item Event-B XText Context (0.0.3):
  \begin{itemize}
  \item Issue \#3: Single-line comment after the element, multi-line comment before
    the element
  \end{itemize}

\item Event-B XText Context IDE (0.0.2): Regenerated
 
\item Event-B XText ContextUI IDE (0.0.2): Regenerated

\item Event-B XText Machine (0.0.3):
  \begin{itemize}
  \item Issue \#3: Single-line comment after the element, multi-line comment before
    the element.

  \item Issue \#5: Event terminator using 'end' keyword instead of ';'
  \end{itemize}

\item Event-B XText Machine IDE (0.0.2) Regenerated
  
\item Event-B XText Machine UI IDE (0.0.2) Regenerated
\end{itemize}


\paragraph{0.0.2}
\begin{itemize}
\item Event-B XText Common (0.0.2):
  \begin{itemize}
  \item Added transient value service for XContext and XMachine.
  \end{itemize}

\item Event-B XText Context (0.0.2):
  \begin{itemize}
  \item Added formatter (used for auto-indentation).
  \end{itemize}

\item Event-B XText Machine (0.0.2):
  \begin{itemize}
  \item Added formatter (used for auto-indentation).
  \end{itemize}

\item Event-B XText UI (0.0.1): Initial version
  \begin{itemize}
  \item Added context menu for converting machines and contexts to XText.
  \end{itemize}
\end{itemize}

\paragraph{0.0.1} Initial version contains the following plug-ins:
\begin{itemize}
\item Event-B XText Branding (0.0.1) Initial version: Branding information

\item Event-B XText Common (0.0.1) Initial version: Common facilities

\item Event-B XText Context (0.0.1) Initial version: Core support for Event-B contexts

\item Event-B XText Context IDE (0.0.1) Initial version: IDE for Event-B contexts

\item Event-B XText Context UI (0.0.1) Initial version: UI for Event-B contexts

\item Event-B XText Machine (0.0.1) Initial version: Core support for Event-B machines

\item Event-B XText Machine IDE (0.0.1) Initial version: IDE for Event-B machines

\item Event-B XText Machine UI (0.0.1) Initial version: UI for Event-B machines
\end{itemize}

\subsubsection{IMPORTANT}
\label{sec:important}

\begin{itemize}
\item Currently, XEvent-B front-end \textbf{not only} supports ``standard'' Event-B machines and contexts, but also supports ``\emph{Machine Inclusion}'' and ``\emph{Event Synchronisation}''.
\item Since the XContexts and XMachines are compiled to the Rodin files, the corresponding Rodin contexts and machines will be \textbf{OVER-WRITTEN}. Any changes in the Rodin files will not be lost.

\item \textbf{DO NOT USE} the XEvent-B Front-end if you use modelling plug-ins such as \emph{iUML-B} state-machines and class-diagrams, as the additional modelling elements will be over-written.

\item Windows users \textbf{must} change the workspace text file encoding to \textbf{UTF-8}. This can be updated under the Rodin Preferences: General/Workspace then in the ``\emph{Text file encoding}'' section, select Other: UTF-8.

\end{itemize}

\subsubsection{Known Issues}
\label{sec:known-issues}

\begin{itemize}
\item Converting to XText: Currently, the ``extended'' attribute of events are not serialised.

\item Machine Inclusion: 
\begin{itemize}
	\item Including the \textbf{same} machine to both the abstract and its refining machine can result in the repetition of invariants.
\end{itemize}

\end{itemize}

\subsubsection{Configuration}
\label{sec:configuration}

\paragraph{Event-B Explorer}
By default, XContext files (extension .bucx) and XMachine files (extension .bumx) are not displayed in the \emph{Event-B Explorer}. To enable this, select \emph{Customize view} for \emph{Event-B Explorer} and uncheck the option \emph{All files and folders}.

\subsection{Basic Tutorial}
\label{sec:basic-tutorial}

This tutorial provides a step-by-step walk-through working with XEvent-B constructs. This tutorial also available as Cheatsheets with the Rodin Platform (\texttt{Help/Cheat Sheets/Event-B XText Cheatsheets/Event-B XText Basic Tutorial}).

\subsubsection{Task 1. Customise the Event-B Explorer}
\textbf{Introduction}
The purpose of this task is to customise the Event-B Explorer so that XEvent-B constructs are visible.
\begin{description}
\item[Step 1. Disable the filter on ``All files and folders''] Select ``Customize View'' of Event-B Explorer View. Make sure that ``All files and folders'' from the dialog is \textbf{Unchecked}.
\end{description}

\textbf{Conclusion} Since the filter on ``All files and folders'' is now disabled, there might be other files and folders than XEvent-B constructs will also be visible in the Event-B Explorer.

\subsubsection{Task 2. Create an Event-B Project}\label{CreateProject}
\textbf{Introduction} The purpose of this task is to create an Event-B project for the XEvent-B constructs. 
\begin{description}
\item[Step 1. Create a new Event-B Project] Create a new Event-B Project named ``Club'' using the \emph{New Event-B Project} wizard (see Figure~\ref{fig:CreateProject}).
\begin{figure}[!htbp]
  \centering
  \ifplastex
  \includegraphics[width=512]{figures/CreateProject}
  \else
  \includegraphics[width=0.9\textwidth]{figures/CreateProject}
  \fi
  \caption{Create Event-B Project called ``Club''}
  \label{fig:CreateProject}
\end{figure}

\end{description}
\textbf{Conclusion} By now, the project ``Club'' should be visible in the Event-B Explorer.


\subsubsection{Task 3. Create a simple XContext coursesCtx.bucx}\label{Sec:SimpleContext}
\textbf{Introduction} The purpose of this task is to create a simple XContext within the newly created project.
\begin{description}
\item[Step 1. Create a new XContext coursesCtx.bucx] Create a new XContext named ``coursesCtx.bucx'' using the \emph{New File wizard} (see Figure~\ref{fig:CreateCoursesCtx}).
         \textbf{Important}: A pop-up dialog will be displayed asking to convert the ``Club''
         project to XText project, please answer \textbf{Yes} (see Figure~\ref{fig:ConvertToXText}).
\begin{figure}[!htbp]
  \centering
  \ifplastex
  \includegraphics[width=512]{figures/CreateCoursesCtx}
  \else
  \includegraphics[width=0.9\textwidth]{figures/CreateCoursesCtx}
  \fi
  \caption{Create an XContext called ``coursesCtx.bucx''}
  \label{fig:CreateCoursesCtx}
\end{figure}
\begin{figure}[!htbp]
  \centering
  \ifplastex
  \includegraphics[width=512]{figures/ConvertToXText}
  \else
  \includegraphics[width=0.9\textwidth]{figures/ConvertToXText}
  \fi
  \caption{Convert ``Club'' to XText project}
  \label{fig:ConvertToXText}
\end{figure}

\item[Step 2. Set the content of courseCtx.bucx] Set the content of ``coursesCtx.bucx'' as follows.
  \begin{center}
    \begin{Bcode}
      \ifplastex
      \Bcontext{} coursesCtx\\
      \Bsets{} CRS\\
      \Bconstants{} m\\
      \Baxioms\\
      @axm0_1: "finite(CRS)"\\
      @axm0_2: "m ∈ ℕ1"\\
      @thm0_1: "0 < m" \Btheorem\\
      \Bend
      \else
\Bcontext{} coursesCtx\\
\Bsets{} CRS\\
\Bconstants{} m\\
\Baxioms\\
\Btab @axm0\_1: "\(\finite(CRS)\)"\\
\Btab @axm0\_2: "\(m \in \natn\)"\\
\Btab @thm0\_1: "\(0 < m\)" \Btheorem\\
\Bend
       \fi
    \end{Bcode}
  \end{center}
  \textbf{Important}: In order to typeset Event-B mathematical symbol, e.g., \ifplastex ℕ1 \else $\natn$ \fi, one can use content assist. For example, typing \texttt{NAT} and invoking content assist (e.g., on Mac OS \texttt{Ctrl+Space}), a dropdown list will appear with options for typesetting \ifplastex ℕ \else $\natn$ \fi and \ifplastex ℕ1 \else $\natn$ \fi (See Figure~\ref{fig:NAT1ContentAssist}.
  \begin{figure}[!htbp]
    \centering
    \ifplastex
    \includegraphics[width=512]{figures/NAT1ContentAssist}
    \else
    \includegraphics[width=0.9\textwidth]{figures/NAT1ContentAssist}
    \fi
    \caption{Type-setting \ifplastex ℕ1 \else $\natn$ \fi using Content Assist}
    \label{fig:NAT1ContentAssist}
  \end{figure}

\item[Step 3. Auto-format the code] Automatically format the content of ``coursesCtx.bucx'' using short-cut (e.g., on Mac OS: \texttt{Cmd+Shift+F}).

\item[Step 4. Save the file] \textbf{Save the file ``coursesCtx.bucx''}.
\end{description}
\textbf{Conclusion} By now, the XContext ``coursesCtx.bucx'' and the corresponding Rodin Context ``coursesCtx'' should be visible in the Event-B Explorer (see Figure~\ref{fig:CoursesCtx}). 
  \begin{figure}[!htbp]
    \centering
    \ifplastex
    \includegraphics[width=512]{figures/CoursesCtx}
    \else
    \includegraphics[width=0.9\textwidth]{figures/CoursesCtx}
    \fi
    \caption{The final XContext coursesCtx.bucx}
    \label{fig:CoursesCtx}
  \end{figure}



\subsubsection{Task 4. Create a simple XMachine m0.bumx}\label{CreateMachine}
\textbf{Introduction} The purpose of this task is to create a simple XMachine within the newly created project.

\begin{description}
\item[Step 1. Create a new XMachine m0.bumx] \textbf{Create a new XMachine} named ``m0.bumx'' using the New File wizard (see Figure~\ref{fig:CreateM0}. The newly created file should be opened automatically in an XMachine editor.
  \begin{figure}[!htbp]
    \centering
    \ifplastex
    \includegraphics[width=512]{figures/CreateM0}
    \else
    \includegraphics[width=0.9\textwidth]{figures/CreateM0}
    \fi
    \caption{Type-setting \ifplastex ℕ1 \else $\natn$ \fi using Content Assist}
    \label{fig:CreateM0}
  \end{figure}

\item[Step 2. Set the content of m0.bumx] \textbf{Set the content of "m0.bumx" as follows}.
  \begin{center}
    \begin{Bcode}
      \ifplastex
      \Bmachine{} m0 \\
      \Bvariables{} crs \\
      \Binvariants \\
      @inv0_1: "crs ∈ ℙ(CRS)"\\
      @thm0_2: "finite(crs)" \Btheorem \\
      @inv0_2: "card(crs) ≤ m" \\
      @DLF: "(card(crs) ≠ m) ∨ (∃cs·cs ⊆ crs ∧ cs ≠ ∅)" \\
      \Bevents\\
      INITIALISATION\\
      \Bbegin \\
      @act0_1: "crs ≔ ∅"\\
      \Bend\\
      OpenCourses\\
      \Bwhen\\
      @grd0_1: "card(crs) ≠ m" \\
      @thm0_3: "crs ≠ CRS" \Btheorem \\
      \Bthen\\
      @act0_1: "crs :∣ crs ⊂ crs' ∧ card(crs') ≤ m"\\
      \Bend\\
      CloseCourses \Banticipated\\
      \Bany{} cs \Bwhere\\
      @grd0_1: "cs ⊆ crs"\\
      @grd0_2: "cs ≠ ∅"\\
      \Bthen\\
      @act0_1: "crs ≔ crs ∖ cs"\\
      \Bend\\
      \Bend
      \else
      \Bmachine{} m0 \\
      \Bvariables{} crs \\
      \Binvariants \\
      \Btab @inv0\_1: "\(crs \in \pow(CRS)\)"\\
      \Btab @thm0\_2: "\(\finite(crs)\)" \Btheorem \\
      \Btab @inv0\_2: "\(\card(crs) \leq m\)" \\
      \Btab @DLF: "\((\card(crs) \neq m) \lor (\exists cs \qdot cs \subseteq crs \land cs \neq \emptyset)\)" \\
      \Bevents\\
      \Btab INITIALISATION\\
      \Btab \Bbegin \\
      \Btab\Btab @act0\_1: "\(crs \bcmeq \emptyset\)"\\
      \Btab \Bend\\
      \Btab OpenCourses\\
      \Btab \Bwhen\\
      \Btab \Btab @grd0\_1: "\(\card(crs) \neq m\)" \\
      \Btab \Btab @thm0\_3: "\(crs \neq CRS\)" \Btheorem \\
      \Btab \Bthen\\
      \Btab \Btab @act0\_1: "\(crs \bcmsuch crs \subset crs' \land \card(crs') \leq m\)"\\
      \Btab \Bend\\
      \Btab CloseCourses \Banticipated\\
      \Btab \Bany{} cs \Bwhere\\
      \Btab \Btab @grd0\_1: "\(cs \subseteq crs\)"\\
      \Btab \Btab @grd0\_2: "\(cs \neq \emptyset\)"\\
      \Btab \Bthen\\
      \Btab \Btab @act0\_1: "\(crs \bcmeq crs \setminus cs\)"\\
      \Btab \Bend\\
      \Bend
      \fi
    \end{Bcode}
  \end{center}
\item[Step 3. Auto-format the code] \textbf{Automatically format the content of ``m0.bumx''} by using short-cut (e.g., on Mac OS: Cmd+Shift+F).

\item[Step 4. Save the file] \textbf{Save the file ``m0.bumx''}.

\item[Step 5. Add missing ``sees'' clause] In the compiled Rodin Machine m0, there are several errors, due to the fact that \textbf{m0} refers to the sets and constants of the context courseCtx.
  \textbf{Add the missing ``sees'' clause} after the ``machine'' clause
  \begin{center}
    \begin{Bcode}
      \Bsees{} courseCtx
    \end{Bcode}
  \end{center}
  (Note: One can use \emph{Content Assist} after typing the ``sees'' keyword to select the context, see Figure~\ref{fig:SeesCoursesCtx}).
  \begin{figure}[!htbp]
    \centering
    \ifplastex
    \includegraphics[width=512]{figures/SeesCoursesCtx}
    \else
    \includegraphics[width=0.9\textwidth]{figures/SeesCoursesCtx}
    \fi
    \caption{Content Assist for adding Sees clause}
    \label{fig:SeesCoursesCtx}
  \end{figure}

\item[Step 6. Save the file again] \textbf{Save the file "m0.bumx" again}.
\end{description}
\textbf{Conclusion} By now, the XMachine ``m0.bumx'' and the corresponding Rodin Machine ``m0'' (without any error) should be visible in the Event-B Explorer (see Figure~\ref{fig:M0}.
  \begin{figure}[!htbp]
    \centering
    \ifplastex
    \includegraphics[width=512]{figures/M0}
    \else
    \includegraphics[width=0.9\textwidth]{figures/M0}
    \fi
    \caption{XMachine m0.bucx}
    \label{fig:M0}
  \end{figure}

\subsubsection{Task 5. Create extended XContexts}
\textbf{Introduction} The purpose of this task is to create some more extended XContexts within the "Club" project.

\paragraph{Task 5.1. Create a simple XContext membersCtx.bucx}
\textbf{Introduction} The purpose of this sub-task is to create a simple XContext ``membersCtx.bucx'' within the ``Club'' project.
\begin{description}
\item[Step 1. Create a new XContext membersCtx.bucx] \textbf{Create a new XContext} named ``membersCtx.bucx'' using the \emph{New File} wizard (see Figure~\ref{fig:CreateMembersCtx}.
  \begin{figure}[!htbp]
    \centering
    \ifplastex
    \includegraphics[width=512]{figures/CreateMembersCtx}
    \else
    \includegraphics[width=0.9\textwidth]{figures/CreateMembersCtx}
    \fi
    \caption{Create membersCtx.bucx}
    \label{fig:CreateMembersCtx}
  \end{figure}

\item[Step 2. Set the content of membersCtx.bucx] \textbf{Set the content of ``membersCtx.bucx'' as follows}.
  \begin{center}
    \begin{Bcode}
      \ifplastex
      \Bcontext{} memebersCtx\\
      \Bsets{} MEM\\
      \Baxioms\\
      @axm0_1: "finite(MEM)"\\
      \Bend
      \else
      \Bcontext{} memebersCtx\\
      \Bsets{} MEM\\
      \Baxioms\\
      \Btab @axm0_1: "\(\finite(MEM)\)"\\
      \Bend
      \fi
    \end{Bcode}
  \end{center}

\item [Step 3. Auto-format the code] \textbf{Automatically format the content of ``membersCtx.bucx''} by using short-cut (e.g., on Mac OS: Cmd+Shift+F).

\item[Step 4. Save the file] <b>Save the file \textbf{``membersCtx.bucx''}.
\end{description}

\textbf{Conclusion} By now, the XContext ``membersCtx.bucx'' and the corresponding Rodin Context ``membersCtx'' should be visible in the Event-B Explorer (see Figure~\ref{fig:membersCtx}.
  \begin{figure}[!htbp]
    \centering
    \ifplastex
    \includegraphics[width=512]{figures/MembersCtx}
    \else
    \includegraphics[width=0.9\textwidth]{figures/MembersCtx}
    \fi
    \caption{XContext membersCtx.bucx}
    \label{fig:membersCtx}
  \end{figure}
\paragraph{Task 5.2. Create an extended XContext participantsCtx.bucx}
\textbf{Introduction} The purpose of this sub-task is to create an extended XContext ``participantsCtx.bucx'' within the ``Club'' project.

\begin{description}
\item[Step 1. Create a new XContext participantsCtx.bucx] \textbf{Create a new XContext} named ``participantsCtx.bucx'' using the \emph{New File wizard} (see Figure~\ref{fig:CreateParticipantsCtx}.
  \begin{figure}[!htbp]
    \centering
    \ifplastex
    \includegraphics[width=512]{figures/CreateParticipantsCtx}
    \else
    \includegraphics[width=0.9\textwidth]{figures/CreateParticipantsCtx}
    \fi
    \caption{Create participantsCtx.bucx}
    \label{fig:CreateParticipantsCtx}
  \end{figure}

\item[Step 2. Set the content of participantsCtx.bucx] \textbf{Set the content of ``participantsCtx.bucx'' as follows}.
  \begin{center}
    \begin{Bcode}
      \ifplastex
      \Bcontext{} participantsCtx\\
      \Bextends{} membersCtx\\
      \Bconstants{} PRTCPT\\
      \Baxioms\\
      @axm1_2: "PRTCPT ∈ ℙ(MEM)"\\
      @thm1_1: "finite(PRTCPT)" \Btheorem\\
      \Bend
      \else
      \Bcontext{} participantsCtx\\
      \Bextends{} membersCtx\\
      \Bconstants{} PRTCPT\\
      \Baxioms\\
      \Btab @axm1_2: "\(PRTCPT \in \pow(MEM)\)"\\
      \Btab @thm1_1: "\(\finite(PRTCPT)\)" \Btheorem\\
      \Bend
      \fi
    \end{Bcode}
  \end{center}

\item[Step 3. Auto-format the code] \textbf{Automatically format the content of ``participantsCtx.bucx''} by using short-cut (e.g., on Mac OS: Cmd+Shift+F).

\item[Step 4. Save the file] \textbf{Save the file ``participantsCtx.bucx''}.
\end{description}

\textbf{Conclusion} By now, the XContext ``participantsCtx.bucx'' and the corresponding Rodin Context ``participantsCtx'' should be visible in the Event-B Explorer (see Figure~\ref{fig:ParticipantsCtx}).
  \begin{figure}[!htbp]
    \centering
    \ifplastex
    \includegraphics[width=512]{figures/ParticipantsCtx}
    \else
    \includegraphics[width=0.9\textwidth]{figures/ParticipantsCtx}
    \fi
    \caption{XContext participantsCtx.bucx}
    \label{fig:ParticipantsCtx}
  \end{figure}

\paragraph{Task 5.3. Create an extended XContext instructorsCtx.bucx}
\textbf{Introduction} The purpose of this sub-task is to create an extended XContext ``instructorsCtx.bucx'' within the ``Club'' project.
\begin{description}
\item[Step 1. Create a new XContext instructorsCtx.bucx] \textbf{Create a new XContext} named ``instructorsCtx.bucx'' using the \emph{New File wizard} (see Figure~\ref{fig:CreateInstructorsCtx}.
  \begin{figure}[!htbp]
    \centering
    \ifplastex
    \includegraphics[width=512]{figures/CreateInstructorsCtx}
    \else
    \includegraphics[width=0.9\textwidth]{figures/CreateInstructorsCtx}
    \fi
    \caption{Create instructorsCtx.bucx}
    \label{fig:CreateInstructorsCtx}
  \end{figure}

\item[Step 2. Set the content of instructorsCtx.bucx] \textbf{Set the content of ``instructorsCtx.bucx'' as follows}.
  \begin{center}
    \begin{Bcode}
      \ifplastex
      \Bcontext{} instructorsCtx\\
      \Bextends{} membersCtx coursesCtx\\
      \Bconstants{} INSTR instrs\\
      \Baxioms\\
      @axm1_3: "INSTR ∈ ℙ(MEM)"\\
      @axm1_4: "instrs ∈ CRS → INSTR"\\
      \Bend
      \else
      \Bcontext{} instructorsCtx\\
      \Bextends{} membersCtx coursesCtx\\
      \Bconstants{} INSTR instrs\\
      \Baxioms\\
      \Btab @axm1_3: "\(INSTR \in \pow(MEM)\)"\\
      \Btab @axm1_4: "\(instrs \in CRS \tfun INSTR\)"\\
      \Bend
      \fi
    \end{Bcode}
  \end{center}

\item[Step 3. Auto-format the code] \textbf{Automatically format the content of ``intructorsCtx.bucx''} by using short-cut (e.g., on Mac OS: Cmd+Shift+F).

\item[Step 4. Save the file] \textbf{Save the file ``instructorsCtx.bucx''}.
\end{description}

\textbf{Conclusion} By now, the XContext ``instructorsCtx.bucx'' and the corresponding Rodin Context ``instructorsCtx'' should be visible in the Event-B Explorer (see Figure.
  \begin{figure}[!htbp]
    \centering
    \ifplastex
    \includegraphics[width=512]{figures/InstructorsCtx}
    \else
    \includegraphics[width=0.9\textwidth]{figures/InstructorsCtx}
    \fi
    \caption{XContext instructorsCtx.bucx}
    \label{fig:instructorsCtx}
  \end{figure}

\subsubsection{Task 6. Create refined XMachines}
\textbf{Introduction} The purpose of this task is to create some more refined XMachines within the ``Club'' project.

\paragraph{Task 6.1. Create a refined XMachine m1.bumx}
\textbf{Introduction} The purpose of this sub-task is to create a refined XMachine ``m1.bumx'' within the ``Club'' project.
\begin{description}
\item[Step 1. Create a new XMachine m1.bumx] \textbf{Create a new XMachine} named ``m1.bumx'' using the \emph{New File wizard} (see Figure~\ref{fig:CreateM1}. The newly created file should be opened automatically in an XMachine editor.
  \begin{figure}[!htbp]
    \centering
    \ifplastex
    \includegraphics[width=512]{figures/CreateM1}
    \else
    \includegraphics[width=0.9\textwidth]{figures/CreateM1}
    \fi
    \caption{Create m1.bumx}
    \label{fig:CreateM1}
  \end{figure}

\item[Step 2. Set the content of m1.bumx] \textbf{Set the content of ``m1.bumx'' as follows}.
  \begin{center}
    \begin{Bcode}
      \ifplastex
      \Bmachine{} m1\\
      \Brefines{} m0\\
      \Bsees{} instructorsCtx participantsCtx \\
      \Bvariables{} crs prtcpts \\
      \Binvariants\\
      @inv1_1: "prtcpts ∈ crs ↔ PRTCPT"\\
      @inv1_2: "∀c·c ∈ crs ⇒ instrs(c) ∉ prtcpts[\{c\}]"\\
      \Bvariant{} "(crs × PRTCPT) ∖ prtcpts"\\
      \Bevents\\
      INITIALISATION \Bextended\\
      \Bbegin\\
      @act1_2: "prtcpts ≔ ∅"\\
      \Bend\\
      OpenCourses \Bextended\\
      \Brefines{} OpenCourses\\
      \Bwhen\\
      @thm1_2: "dom(prtcpts) ⊆ crs" theorem \\
      \Bend\\
      CloseCourses \Bextended{} \Banticipated\\
      \Brefines{} CloseCourses\\
      \Bbegin\\
      @act1_2: "prtcpts ≔ cs ⩤ prtcpts"\\
      \Bend\\
      Register \Bconvergent\\
      \Bany{} p c \Bwhere \\
      @grd1_1: "p ∈ PRTCPT"\\
      @grd1_2: "c ∈ crs"\\
      @grd1_3: "p ≠ instrs(c)"\\
      @grd1_4: "c ↦ p ∉ prtcpts"\\
      \Bthen\\
      @act1_1: "prtcpts ≔ prtcpts ∪ \{c ↦ p\}"\\
      \Bend\\
      \Bend
      \else
      \Bmachine{} m1\\
      \Brefines{} m0\\
      \Bsees{} instructorsCtx participantsCtx \\
      \Bvariables{} crs prtcpts \\
      \Binvariants\\
      \Btab @inv1_1: "\(prtcpts \in crs \rel PRTCPT\)"\\
      \Btab @inv1_2: "\(\forall c \qdot c ∈ crs \limp instrs(c) \notin prtcpts[\{c\}]\)"\\
      \Bvariant{} "\((crs \cprod PRTCPT) \setminus prtcpts\)"\\
      \Bevents\\
      \Btab INITIALISATION \Bextended\\
      \Btab \Bbegin\\
      \Btab \Btab @act1_2: "\(prtcpts \bcmeq \emptyset\)"\\
      \Btab \Bend\\
      \Btab OpenCourses \Bextended\\
      \Btab \Brefines{} OpenCourses\\
      \Btab \Bwhen\\
      \Btab \Btab @thm1_2: "\(\dom(prtcpts) \subseteq crs\)" theorem \\
      \Btab \Bend\\
      \Btab CloseCourses \Bextended{} \Banticipated\\
      \Btab \Brefines{} CloseCourses\\
      \Btab \Bbegin\\
      \Btab \Btab @act1_2: "\(prtcpts \bcmeq cs \domsub prtcpts\)"\\
      \Btab \Bend\\
      \Btab Register \Bconvergent\\
      \Btab \Bany{} p c \Bwhere \\
      \Btab \Btab @grd1_1: "\(p \in PRTCPT\)"\\
      \Btab \Btab @grd1_2: "\(c \in crs\)"\\
      \Btab \Btab @grd1_3: "\(p \neq instrs(c)\)"\\
      \Btab \Btab @grd1_4: "\(c \mapsto p \neq prtcpts\)"\\
      \Btab \Bthen\\
      \Btab \Btab @act1_1: "\(prtcpts \bcmeq prtcpts \bunion \{c \mapsto p\}\)"\\
      \Btab \Bend\\
      \Bend
      \fi
    \end{Bcode}
  \end{center}

\item[Step 3. Auto-format the code] \textbf{Automatically format the content of ``m1.bumx''} by using short-cut (e.g., on Mac OS: Cmd+Shift+F).

\item[Step 4. Save the file] Save the file ``m1.bumx''.
\end{description}

\textbf{Conclusion} By now, the XMachine ``m1.bucx'' and the corresponding Rodin Machine ``m1'' should be visible in the Event-B Explorer (see Figure~\ref{fig:M1}.
  \begin{figure}[!htbp]
    \centering
    \ifplastex
    \includegraphics[width=512]{figures/M1}
    \else
    \includegraphics[width=0.9\textwidth]{figures/M1}
    \fi
    \caption{XMachine m1.bumx}
    \label{fig:M1}
  \end{figure}

\paragraph{Task 6.2. Create a refined XMachine m2.bumx}
\textbf{Introduction} The purpose of this sub-task is to create a refined XMachine ``m2.bumx'' within the ``Club'' project.

\begin{description}
\item[Step 1. Create a new XMachine m2.bumx] \textbf{Create a new XMachine} named ``m2.bumx'' using the \emph{New File wizard} (see Figure~\ref{fig:CreateM2}. The newly created file should be opened automatically in an XMachine editor.
  \begin{figure}[!htbp]
    \centering
    \ifplastex
    \includegraphics[width=512]{figures/CreateM2}
    \else
    \includegraphics[width=0.9\textwidth]{figures/CreateM2}
    \fi
    \caption{Create m2.bumx}
    \label{fig:CreateM2}
  \end{figure}

\item[Step 2. Set the content of m2.bumx] \textbf{Set the content of ``m2.bumx'' as follows}.
  \begin{center}
    \begin{Bcode}
      \ifplastex
      \Bmachine{} m2\\
      \Brefines{} m1\\
      \Bsees{} instructorsCtx participantsCtx\\
      \Bvariables{} atnds\\
      \Binvariants\\
      @inv2_1: "atnds ∈ CRS ⇸ ℙ(PRTCPT)"\\
      @inv2_2: "crs = dom(atnds)"\\
      @inv2_3: "∀c·c ∈ crs ⇒ prtcpts[\{c\}] = atnds(c)"\\
      @thm2_1: "finite(atnds)" \Btheorem\\
      \Bvariant{} "card(atnds)"\\
      \Bevents\\
      INITIALISATION\\
      \Bbegin\\
      @act2_1: "atnds ≔ ∅"\\
      \Bend\\
      OpenCourse\\
      \Brefines{} OpenCourses\\
      \Bany{} c \Bwhere\\
      @grd2_1: "c ∉ dom(atnds)"\\
      @grd2_2: "card(atnds) ≠ m" \\
      @thm2_2: "card(crs) ≠ m" theorem\\
      \Bwith\\
      @crs': "crs' = crs ∪ \{c\}"\\
      \Bthen\\
      @act2_1: "atnds(c) ≔ ∅"\\
      \Bend\\
      CloseCourse \Bconvergent\\
      \Brefines{} CloseCourses\\
      \Bany{} c \Bwhere\\
      @grd2_1: "c ∈ dom(atnds)"\\
      \Bwith\\
      @cs: "cs = \{c\}"\\
      \Bthen\\
      @act1_2: "atnds ≔\{c\} ⩤ atnds"\\
      \Bend\\
      Register \Bconvergent\\
      \Brefines{} Register\\
      \Bany{} p c \Bwhere\\
      @grd2_1: "p ∈ PRTCPT"\\
      @grd2_2: "p ≠ instrs(c)"\\
      @grd2_3: "c ∈ dom(atnds)"\\
      @grd2_4: "p ∉ atnds(c)"\\
      @thm2_3: "atnds(c) = prtcpts[\{c\}]" theorem\\
      \Bthen\\
      @act2_1: "atnds(c) ≔ atnds(c) ∪ \{p\}"\\
      \Bend\\
      \Bend
      \else
      \Bmachine{} m2\\
      \Brefines{} m1\\
      \Bsees{} instructorsCtx participantsCtx\\
      \Bvariables{} atnds\\
      \Binvariants\\
      \Btab @inv2_1: "\(atnds \in CRS \pfun \pow(PRTCPT)\)"\\
      \Btab @inv2_2: "\(crs = \dom(atnds)\)"\\
      \Btab @inv2_3: "\(\forall c \qdot c \in crs \limp prtcpts[\{c\}] = atnds(c)\)"\\
      \Btab @thm2_1: "\(\finite(atnds)\)" \Btheorem\\
      \Bvariant{} "\(\card(atnds)\)"\\
      \Bevents\\
      \Btab INITIALISATION\\
      \Btab \Bbegin\\
      \Btab \Btab @act2_1: "\(atnds \bcmeq \emptyset\)"\\
      \Btab \Bend\\
      \Btab OpenCourse\\
      \Btab \Brefines{} OpenCourses\\
      \Btab \Bany{} c \Bwhere\\
      \Btab \Btab @grd2_1: "\(c \notin \dom(atnds)\)"\\
      \Btab \Btab @grd2_2: "\(\card(atnds) \neq m\)" \\
      \Btab \Btab @thm2_2: "\(\card(crs) \neq m\)" \Btheorem\\
      \Btab \Bwith\\
      \Btab \Btab @crs': "\(crs' = crs \bunion \{c\}\)"\\
      \Btab \Bthen\\
      \Btab \Btab @act2_1: "\(atnds(c) \bcmeq \emptyset\)"\\
      \Btab \Bend\\
      \Btab CloseCourse \Bconvergent\\
      \Btab \Brefines{} CloseCourses\\
      \Btab \Bany{} c \Bwhere\\
      \Btab \Btab @grd2_1: "\(c \in \dom(atnds)\)"\\
      \Btab \Bwith\\
      \Btab \Btab @cs: "\(cs = \{c\}\)"\\
      \Btab \Bthen\\
      \Btab \Btab @act1_2: "\(atnds \bcmeq \{c\} \domsub atnds\)"\\
      \Btab \Bend\\
      \Btab  Register \Bconvergent\\
      \Btab \Brefines{} Register\\
      \Btab \Bany{} p c \Bwhere\\
      \Btab \Btab @grd2_1: "\(p \in PRTCPT\)"\\
      \Btab \Btab @grd2_2: "\(p \neq instrs(c)\)"\\
      \Btab \Btab @grd2_3: "\(c \in \dom(atnds)\)"\\
      \Btab \Btab @grd2_4: "\(p \notin atnds(c)\)"\\
      \Btab \Btab @thm2_3: "\(atnds(c) = prtcpts[\{c\}]\)" theorem\\
      \Btab \Bthen\\
      \Btab \Btab @act2_1: "\(atnds(c) \bcmeq atnds(c) ∪ \{p\}\)"\\
      \Btab \Bend\\
      \Bend
      \fi
    \end{Bcode}
  \end{center}

\item[Step 3. Auto-format the code] \textbf{Automatically format the content of ``m2.bumx''} by using short-cut (e.g., on Mac OS: Cmd+Shift+F).

\item[Step 4. Save the file] \textbf{Save the file ``m2.bumx''}.
\end{description}
\textbf{Conclusion} By now, the XMachine ``m2.bucx'' and the corresponding Rodin Machine ``m2'' should be visible in the Event-B Explorer (see Figure~\ref{fig:M2}.
  \begin{figure}[!htbp]
    \centering
    \ifplastex
    \includegraphics[width=512]{figures/M2}
    \else
    \includegraphics[width=0.9\textwidth]{figures/M2}
    \fi
    \caption{XMachine m2.bumx}
    \label{fig:M2}
  \end{figure}

\subsection{Advanced Tutorial}
\label{sec:advanced-tutorial}

This tutorial provides a step-by-step walk-through working with \emph{machine inclusion} using XEvent-B. Following the same steps as in Section~\ref{sec:basic-tutorial} to create machines and contexts, we can create a machine that can include other machines and can update the included machines variables via \emph{event synchronisation}.

We illustrate the application of machine inclusion using XEvent-B by modelling a small example of ``controlling cars on a bridge'', which is based on Chapter 2 of ``\emph{Modeling in Event-B: System and Software Engineering}'' book.

\subsubsection{Task 1. Create the reusable model}
\textbf{Introduction} The purpose of this task is to create the model that will be reused by other models using machine inclusion.
\begin{description}
\item[Step 1. Create a new Project (Sensor) with XMachine m0\_SNSR.bumx] 

Following the same steps as 
\ifplastex
in tasks 2 and 4 in Section~\ref{sec:basic-tutorial} for creating project and machines.
\else
in Sections~\ref{CreateProject} and~\ref{CreateMachine} for creating project and machines.
\fi

\item[Step 2. Set the content of m0\_SNSR.bumx] \textbf{Set the content of ``m0\_SNSR.bumx'' as follows}.
\begin{center}
	\begin{Bcode}
		\ifplastex
		\Bmachine{} m0_SNSR\\
		\Bvariables{} SNSR\\
		\Binvariants\\
		@thm0_1: "SNSR ∈ BOOL" \Btheorem\\
		\Bevents\\
		INITIALISATION\\
		\Bbegin\\
		@act1: "SNSR ≔ FALSE"\\
		\Bend\\
		SNSR_on\\
		\Bwhen\\
		@grd1: "SNSR = FALSE"\\
		\Bthen\\
		@act1: "SNSR ≔ TRUE"\\
		\Bend\\
	    SNSR_off\\
	    \Bwhen\\
	    @grd1: "SNSR = TRUE"\\
	    \Bthen\\
	    @act1: "SNSR ≔ FALSE"\\
	    \Bend\\
		\Bend
		\else
		\Bmachine{} m0_SNSR\\
		\Bvariables{} SNSR\\
		\Binvariants\\
		\Btab @thm0_1: "\(SNSR \in BOOL\)" \Btheorem\\
		\Bevents\\
		\Btab INITIALISATION\\
		\Btab \Bbegin\\
		\Btab \Btab @act1: "\(SNSR \bcmeq FALSE\)"\\
		\Btab \Bend\\
		\Btab SNSR_on\\
		\Btab \Bwhen\\
		\Btab \Btab @grd1: "\(SNSR = FALSE\)"\\
		\Btab \Bthen\\
		\Btab \Btab @act1: "\(SNSR \bcmeq TRUE\)"\\
		\Btab \Bend\\
		\Btab SNSR_off\\
		\Btab \Bwhen\\
		\Btab \Btab @grd1: "\(SNSR = TRUE\)"\\
		\Btab \Bthen\\
		\Btab \Btab @act1: "\(SNSR \bcmeq FALSE\)"\\
		\Btab \Bend\\
		\Bend
		\fi
	\end{Bcode}
\end{center}

\item[Step 3. Auto-format and Save the file ``m0\_SNSR.bumx''] 
\end{description}

\textbf{Conclusion} By now, the XMachine ``m0\_SNSR.bumx'' and the corresponding Rodin Machine ``m0\_SNSR'' should be visible in the Event-B Explorer.

\subsubsection{Task 2. Model the abstract level of cars on a bridge}
\textbf{Introduction} The purpose of this task is to create the abstract model of the ``cars on a bridge'' example. At this level, we have not applied machine inclusion, but it is possible to apply machine inclusion right from the abstract level. 
\begin{description}
	\item[Step 1. Create the Context c0\_limit.bucx in a new project Car] 
	Following the same steps as 
	\ifplastex
	in task 3 of Section~\ref{sec:basic-tutorial} for creating a simple context.  
	\else
	in Section~\ref{Sec:SimpleContext} for creating a simple context.
	\fi
	\textbf{Set the content of ``c0\_limit.bucx'' as follows and save the file}.
	  
	  \begin{center}
		\begin{Bcode}
			\ifplastex
			\Bcontext{} c0_limit\\
			\Bconstants{} D\\
			\Baxioms\\
			@axm1: "D ∈ ℕ1"\\
			\Bend
			\else
			\Bcontext{} c0_limit\\
			\Bconstants{} D\\
			\Baxioms\\
			\Btab @axm1: "\(D \in \nat1\)"\\
			\Bend
			\fi
		\end{Bcode}
	\end{center}
	\item[Step 2. Create the Machine m0\_cars.bumx]\textbf{Set the content of ``m0\_cars.bumx'' as follows and save the file}.

	\begin{center}
		\begin{Bcode}
			\ifplastex
			\Bmachine{} m0_cars\\
			\Bsees{} c0_limit\\
			\Bvariables{} A B C\\
			\Binvariants\\
			@inv0_1: "A ∈ ℕ"\\
			@inv0_2: "B ∈ ℕ"\\
			@inv0_3: "C ∈ ℕ"\\
			@inv0_4: "A = 0 ∨ C = 0"\\
			@inv0_5: "A + B + C ≤ D"\\
			@thm0_1: "B ≤ D" \Btheorem\\
			\Bevents\\
			INITIALISATION\\
			\Bbegin\\
			@act1: "A ≔ 0"\\
			@act2: "B ≔ 0"\\
			@act3: "C ≔ 0"\\
			\Bend\\
			ML_out\\
			\Bwhen\\
			@grd1: "C = 0"\\
			@grd2: "A + B ≠ D"\\
			\Bthen\\
			@act1: "A ≔ A + 1"\\
			\Bend\\
			ML_in\\
			\Bwhen\\
			@grd1: "C ≠ 0"\\
			\Bthen\\
			@act1: "C ≔ C − 1"\\
			\Bend\\
			IL_in\\
			\Bwhen\\
			@grd1: "A ≠ 0"\\
			\Bthen\\
			@act1: "A ≔ A − 1"\\
			@act2: "B ≔ B + 1"\\
			\Bend\\
			IL_out\\
			\Bwhen\\
			@grd1: "B ≠ 0"\\
			@grd2: "A = 0"\\
			\Bthen\\
			@act1: "B ≔ B − 1"\\
			@act2: "C ≔ C + 1"\\
			\Bend\\
			\Bend
			\else
			\Bmachine{} m0_cars\\
			\Bsees{} c0_limit\\
			\Bvariables{} A B C\\
			\Binvariants\\
			\Btab @inv0_1: "\(A \in \nat\)"\\
			\Btab @inv0_2: "\(B \in \nat"\)\\
			\Btab @inv0_3: "\(C \in \nat\)"\\
			\Btab @inv0_4: "\(A = 0 \vee C = 0\)"\\
			\Btab @inv0_5: "\(A + B + C \leq D\)"\\
			\Btab @thm0_1: "\(B \leq D\)" \Btheorem\\
			\Bevents\\
			\Btab INITIALISATION\\
			\Btab \Bbegin\\
			\Btab \Btab @act1: "\(A \bcmeq 0\)"\\
			\Btab \Btab @act2: "\(B \bcmeq 0\)"\\
			\Btab \Btab @act3: "\(C \bcmeq 0\)"\\
			\Btab \Bend\\
			\Btab ML_out\\
			\Btab \Bwhen\\
			\Btab \Btab @grd1: "\(C = 0\)"\\
			\Btab \Btab @grd2: "\(A + B \neq D\)"\\
			\Btab \Bthen\\
			\Btab \Btab @act1: "\(A \bcmeq A + 1\)"\\
			\Btab \Bend\\
			\Btab ML_in\\
			\Btab \Bwhen\\
			\Btab \Btab @grd1: "\(C \neq 0\)"\\
			\Btab \Bthen\\
			\Btab \Btab @act1: "\(C \bcmeq C - 1\)"\\
			\Btab \Bend\\
			\Btab IL_in\\
			\Btab \Bwhen\\
			\Btab \Btab @grd1: "\(A \neq 0\)"\\
			\Btab \Bthen\\
			\Btab \Btab @act1: "\(A \bcmeq A - 1\)"\\
			\Btab \Btab @act2: "\(B \bcmeq B + 1\)"\\
			\Btab \Bend\\
			\Btab IL_out\\
			\Btab \Bwhen\\
			\Btab \Btab @grd1: "\(B \neq 0\)"\\
			\Btab \Btab @grd2: "\(A = 0\)"\\
			\Btab \Bthen\\
			\Btab \Btab @act1: "\(B \bcmeq B - 1\)"\\
			\Btab \Btab @act2: "\(C \bcmeq C + 1\)"\\
			\Btab \Bend\\
			\Bend
			\fi
		\end{Bcode}
	\end{center}
	
\end{description}
\textbf{Conclusion} Saving the XContext and XMachine files will generate the corresponding Rodin files. In the ``Car'' you have the context ``c0\_limit'' and the machine ``m0\_cars''. Ideally the reusable models should be in a different project, that is why we added the reusable model in a different project ``Sensor''.

\subsubsection{Task 3. Model an XMachine using machine inclusion}
\textbf{Introduction} In this task we define the XMachine ``m1\_SNSR.bumx'' which is a refinement of the machine ``m0\_cars'' and includes two instances of ``m0\_SNSR''. The keywords in red  are \textbf{not} part of the standard Event-B syntax, they correspond to machine inclusion and event synchronisation. 

\begin{description}
	\item[Step 1. Create the file ``m1\_SNSR.bumx''] \textbf{Set its contents as follows.}
	
		\begin{center}
		\begin{Bcode}
			\ifplastex
			\Bmachine{} Car_m1_SNSR\\
			\textcolor{red}{\textbf{includes}} Sensor.m0_SNSR \textcolor{red}{\textbf{as}} IL_out ML_out\\
			\Brefines{} m0_cars\\
			\Bsees{} c0_limit\\
			\Bvariables{} A B C\\
			\Binvariants\\
			@inv1_1: "IL_out_SNSR = TRUE ⇒ B ≠ 0"\\
			\Bevents\\
			INITIALISATION \Bextended\\
			\textcolor{red}{\textbf{synchronises}} IL_out.INITIALISATION\\
			\textcolor{red}{\textbf{synchronises}} ML_out.INITIALISATION\\
			\Brefines{} INITIALISATION\\
			\Bend\\
			ML_out \Bextended\\
			\textcolor{red}{\textbf{synchronises}} ML_out.SNSR_off\\
			\Brefines{} ML_out\\
			\Bend\\
			ML_in \Bextended\\
			\Brefines{} ML_in\\
			\Bend\\
			IL_in \Bextended\\
			\Brefines{} IL_in\\
			\Bwhen\\
			@inv0_2−copy: "B ∈ ℕ" \Btheorem\\
			\Bend\\
			IL_out\\
			\textcolor{red}{\textbf{synchronises}} IL_out.SNSR_off\\
			\Brefines{} IL_out\\
			\Bwhen\\
			@grd2: "A = 0"\\
			\Bthen\\
			@act1: "B ≔ B − 1"\\
			@act2: "C ≔ C + 1"\\
			\Bend\\
			ML_out_ARR\\
			\textcolor{red}{\textbf{synchronises}} ML_out.SNSR_on\\
			\Bend\\
			IL_out_ARR\\
			\textcolor{red}{\textbf{synchronises}} IL_out.SNSR_on\\
			\Bwhen\\
			@grd2: "B ≠ 0"\\
			\Bend\\
			\Bend
			\else
			\Bmachine{} m1_SNSR\\
			\textcolor{red}{includes} Sensor.m0_SNSR \textcolor{red}{as} IL_out ML_out\\
			\Brefines{} m0_cars\\
			\Bsees{} c0_limit\\
			\Bvariables{} A B C\\
			\Binvariants\\
			\Btab @inv1_1: "\("IL_out_SNSR = TRUE \Rightarrow B \neq 0"\)"\\
			\Bevents\\
			\Btab INITIALISATION \Bextended\\
			\Btab \textcolor{red}{synchronises} IL_out.INITIALISATION\\
            \Btab \textcolor{red}{synchronises} ML_out.INITIALISATION\\
            \Btab \Brefines{} INITIALISATION\\
			\Btab \Bend\\
			\Btab ML_out \Bextended\\
			\Btab \textcolor{red}{synchronises} ML_out.SNSR_off\\
			\Btab \Brefines{} ML_out\\
			\Btab \Bend\\
			\Btab ML_in \Bextended\\
			\Btab \Brefines{} ML_in\\
			\Btab \Bend\\
			\Btab IL_in \Bextended\\
			\Btab \Brefines{} IL_in\\
			\Btab \Bwhen\\
			\Btab \Btab @inv0_2−copy: "\(B \in \nat\)" \Btheorem\\
			\Btab \Bend\\
			\Btab IL_out\\
			\Btab \textcolor{red}{synchronises} IL_out.SNSR_off\\
			\Btab \Brefines{} IL_out\\
			\Btab \Bwhen\\
			\Btab \Btab @grd2: "\(A = 0\)"\\
			\Btab \Bthen\\
			\Btab \Btab @act1: "\(B \bcmeq B - 1\)"\\
			\Btab \Btab @act2: "\(C \bcmeq C + 1\)"\\
			\Btab \Bend\\
			\Btab ML_out_ARR\\
			\Btab \textcolor{red}{synchronises} ML_out.SNSR_on\\
			\Btab \Bend\\
			\Btab IL_out_ARR\\
			\Btab \textcolor{red}{synchronises} IL_out.SNSR_on\\
			\Btab \Bwhen\\
			\Btab \Btab @grd2: "\(B \neq 0\)"\\
			\Btab \Bend\\
			\Bend
			\fi
		\end{Bcode}
	\end{center}
	\item[Step 2. Auto-format the file ``m1\_SNSR.bumx'' and Save it.]
\end{description}
\textbf{Conclusion} After saving the file a standard Event-B machine ``m1\_SNSR''will be generated. The generated machine (Figure~\ref{fig:FlattenedMachine}) is flattened to include the variables and invariants  of the included machine ``m0\_SNSR'' which are renamed according to the chosen prefixes. In addition to the guards and actions of the synchronised events. The project name must be specified when including a machine (e.g., Sensor.m0\_SNSR), and the project (Sensor) of the included machines must be opened in the same workspace. You can also use content assist to see all available machines in the workspace.

When synchronising an event you can add the prefix of the required machine followed by the synchronised event name (e.g., IL\_out.SNSR\_on where ``IL\_out'' is one of the included machine prefixes and ``SNSR\_on'' is the synchronised event). It is also possible to include more than one machine and synchronise with more than one event. Notice the order of the generated elements in the flattened machine is the included elements from last to first then the source machine elements.

\begin{figure}[!htbp]
	\centering
	\ifplastex
	\includegraphics[width=512]{figures/Flattened_var_m1_snsr}
	\else
	\includegraphics[width=0.9\textwidth]{figures/Flattened_var_m1_snsr}
	\fi
	\caption{Flattened Machine ``m1\_SNSR''}
	\label{fig:FlattenedMachine}
\end{figure}

% probably here add a figure of the flattened machine to show how variables, inv, gurads are copied and renamed

%%% Local Variables:
%%% mode: latex
%%% TeX-master: "user_manual"
%%% End:
